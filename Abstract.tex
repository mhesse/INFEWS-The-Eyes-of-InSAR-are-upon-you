\documentclass[12pt]{article}
%

% Set font to Helvetica
\usepackage[T1]{fontenc}
\usepackage[utf8]{inputenc} % input umlauts
%\usepackage
\usepackage{lmodern}
\renewcommand{\familydefault}{\sfdefault}
%\usepackage{times} 

% Set page margins
\usepackage[margin=1in]{geometry}

% Set double linespacing
\usepackage{setspace}
%\doublespacing
\singlespacing

% Figures
\usepackage{graphicx}
\graphicspath{ {Figures/} }
\usepackage{wrapfig}

% Equations
\usepackage{amsmath, amssymb, bm}

% Fancy colors - that can be called by names (https://en.wikibooks.org/wiki/LaTeX/Colors)
\usepackage[usenames,dvipsnames]{color}

% Hyperrefs
\usepackage[plainpages=false, colorlinks=true,
  citecolor=BlueViolet, filecolor=black, linkcolor=RoyalPurple,
  urlcolor=RoyalPurple]{hyperref}

% Reference style
\usepackage{natbib}

% Hypotheses
\usepackage{ntheorem}
\theoremseparator{:}
\newtheorem{hyp}{Hypothesis} 
\newcounter{subhyp} 
\newcommand{\subhyp}{ 
  \setcounter{subhyp}{0} 
  \renewcommand\thehyp{\protect\stepcounter{subhyp}% 
  \arabic{hyp}\alph{subhyp}\protect\addtocounter{hyp}{-1}} 
} 
\newcommand{\normhyp}{ 
  \renewcommand\thehyp{\arabic{hyp}} 
  \stepcounter{hyp} 
} 

% Definition of new commands
\newcommand{\note}[1]{\textcolor{red}{#1}}
\newcommand{\sepu}{\vspace{10pt} \hrule \vspace{6pt}\noindent}
\newcommand{\sepl}{\vspace{6pt} \hrule \vspace{10pt}}
\newcommand{\Al}{$^{26}$Al}
\newcommand{\degC}{$^{\circ}$C}
% Title, author, date
\title{Mulitphase dynamics of planetesimal differentiation}
\author{Marc A. Hesse}

\begin{document}

\maketitle
\begin{abstract}
Geochemical and geophysical observations place tight constrains on the time scales of core formation withing the earliest planetesimals.
The physical processes that allow metal-silicate differentiation on planetesimals in the early solar system are not well understood. Metal-silicate differentiation occurs in the presence of two melt phases with opposing buoyancy, either in a compacting porous medium or in a magma ocean. Depending on the thermal structure both process can act within the same body either at different time and/or at different depth. Metal-silicate differentiation of primordial planetesimal therefore involves complex multi-phase flows.
Here we focus on the physics of the segregation of metal and silicate melts in a compacting porous medium. The dynamics of this process involve a complex coupling across the scales. The large-scale redistribution of energy and mass depends on pore-scale interactions between the two melts. Therefore, we propose the development of novel a multi-scale model, where the constitutive laws for the macroscopic evolution equations are determined by pore-scale simulations.
\end{abstract}
\end{document}