% Satellite Missions
In order to better observe freshwater resources, recent satellite missions have provided massive amounts of data, such as precipitation from the Tropical Rainfall Measurement Mission (TRMM) and the Global Precipitation Measurement (GPM) mission, surface soil moisture from the Soil Moisture Active and Passive (SMAP) mission, total water storage change from the Gravity Recovery and Climate Experiment (GRACE) and the GRACE Follow-on (GRACE-FO) mission. In particular, surface deformation data derived from interferometric synthetic aperture radar (InSAR) (1992-present) can be used for characterizing groundwater levels and storage properties in confined aquifers, the critical metric needed for effective groundwater management. While spaceborne Earth-observing missions have provided large amounts of data that can be used to study freshwater resources, these data have been greatly under-utilized in operational water management practices. A grand challenge we need to address today is how to assimilate satellite and in-situ data into quantitative, predictive and computational groundwater models to better assist decision making.

% 