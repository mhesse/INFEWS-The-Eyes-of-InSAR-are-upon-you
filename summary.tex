
% \documentclass[11pt,final]{siamltex}
\documentclass[11pt,final]{article}
%\usepackage[round]{natbib}
\usepackage[square,numbers,sort]{natbib}
\usepackage{graphics}
\usepackage{amsmath,amssymb,mathrsfs}
\usepackage[usenames,dvipsnames]{color}
\usepackage[pdftex]{graphicx}
\usepackage{wrapfig}
\usepackage{url}
\usepackage[small,bf,up]{caption}
\renewcommand{\captionfont}{\footnotesize}
\usepackage[left=1in,right=1in,top=1in,bottom=1in]{geometry}
\usepackage{url}
% \usepackage[sf,bf,small,compact]{titlesec}
 \usepackage[sf,bf,small]{titlesec}

\titleformat{\subsection}[runin]
               {\normalfont\bfseries}
               {\thesubsection.}{.5em}{}[.]
\def\CO2{CO$_2$}
%% \setlength{\oddsidemargin}{0in}
%% \setlength{\evensidemargin}{0in}
%% \setlength{\textwidth}{6.5in}
%% %\setlength{\topmargin}{48pt}
%% \setlength{\topmargin}{0pt}
%% \setlength{\headheight}{0in}
%% \setlength{\headsep}{0in}
%% \setlength{\textheight}{9in}
%% To get the default sans serif font in latex, uncomment following
%% line:
%% Note: the default sf font is much nicer than ariel/helvetica
\renewcommand*\familydefault{\sfdefault}

\def\squeeze{\parskip=0pt\itemsep=0pt}
\let\itemOld=\item
\def\item{\squeeze\itemOld}

%% \def\Heading#1{\par\bigskip\noindent\centerline{\bf\large #1}\nobreak\par}
%% \def\Subheading#1{\par\bigskip\noindent\centerline{\bf #1}\medskip}
%% \def\Subsubheading#1{\par\medskip\noindent{\bf #1.}}

%% \newcounter{sect}
%% \setcounter{sect}{0}
%% \def\Section#1{\addtocounter{sect}{1}\setcounter{subsect}{0}%
%% \par\medskip\noindent\centerline{\bf\large\thesect. #1}\nobreak\par\nobreak}

%% \newcounter{subsect}
%% \def\Subsection#1{\addtocounter{subsect}{1}%
%% \par\medskip\noindent{\bf\thesect.\thesubsect. #1.}}

% \def\Paragraph#1{\par{\bf #1.}}

 \usepackage[plainpages=false, colorlinks=true,
   citecolor=BlueViolet, filecolor=black, linkcolor=RoyalPurple,
   urlcolor=RoyalPurple]{hyperref}


%\renewcommand{\baselinestretch}{0.99}

\newenvironment{nilList}
{\begin{list}{}{\partopsep=4pt plus 2pt minus 2pt
\topsep=0pt\labelwidth=10pt\labelsep=8pt\leftmargin=20pt
\rightmargin=0pt\itemsep=0pt\itemindent=-10pt}}
{\end{list}}

\newcommand{\sig}{\boldsymbol\sigma}
\newcommand{\lam}{\boldsymbol\lambda}
\newcommand{\q}{\mathbf{q}}
\newcommand{\g}{\mathbf{g}}
\newcommand{\x}{\mathbf{x}}
\newcommand{\D}{\mathbf{D}}
\newcommand{\I}{\mathbf{I}}
\newcommand{\K}{\mathbf{K}}
\newcommand{\Stor}{S_{\epsilon}}
\renewcommand{\u}{\ensuremath{\mathbf{u}}}
\newcommand{\s}{\boldsymbol\sigma}
\newcommand{\f}{\ensuremath{\mathbf{f}}}

\newcommand{\note}[1]{\textcolor{red}{ #1}}
\newcommand{\todo}[1]{\textcolor{red}{ \bf #1}}
\newcommand{\sepu}{\vspace{10pt} \hrule \vspace{6pt}\noindent}
\newcommand{\sepl}{\vspace{6pt} \hrule \vspace{10pt}}

\renewcommand{\matrix}[1] {\ensuremath{\boldsymbol{#1}}}
\renewcommand{\vec}[1] {\ensuremath{\boldsymbol{#1}}}

\renewcommand{\citep}{\cite}
\renewcommand{\citet}{\cite}

\begin{document}

\pagestyle{empty}

%##############################################################################
% Project Summary

%\centerline{\Large\bf }
%\Subsubheading{Project summary}
%\Subsubheading{Intellectual merit}
%\Subsubheading{Broader impacts}



%##############################################################################
%\newpage\setcounter{page}{1}
\begin{center}
%{\Large\bf A Bayesian computational framework for characterization
%  and management of groundwater resources via integration of
%  large-scale satellite geodetic data and poromechanics models}
{\bf Project Summary}
\end{center}

%%%%%%%%%%%%%%   V I S I O N  %%%%%%%%%%%%%%%%%%%%%

{\bf Overview.}  Over the last decade, the production of coal and its
share in the world energy supply have increased dramatically even in
countries with large investments in renewable energies.  In the next
two decades, several hundred million people will be lifted out of
poverty and have access to electricity for the first time. If current
trends continue, most of this electricity will come from coal.  This
explains the critical need to develop the technology for sustainable Carbon
Capture and Storage (CCS). CCS has the potential for significant
reductions of anthropogenic \CO2 emissions, because deep saline
aquifers provide large storage volumes. However, geological carbon
storage faces two main challenges, namely the risk to induce
seismicity leading to property damage and loss of life, and leakage of
the injected \CO2 into potable aquifers.  The characterization of the
injection site and continued monitoring of the \CO2 migration as well
as stress changes in the region of elevated pressure are therefore
particularly important to maximize the amount of \CO2 that can be
stored, while ensuring the long term safety of storage sites.

The overall goal of our proposed research is thus to (1) integrate
well pressure and, where available, surface deformation data into
coupled poromechanics models by solving the {\em inverse problem} for
unknown subsurface properties; (2) to quantify the uncertainty in the
inference of the subsurface properties, and (3) to use the resulting
inferred poromechanics models together with their uncertainty to
design {\em optimal control} strategies for well injection that
optimize the amount of stored \CO2 while controlling the risk of
human-induced seismicity.  {\em It is essential that this
  poromechanics-based inference/prediction/control framework takes
  into account uncertainties at every stage, since both the
  observational data and the models are uncertain.} However, solving
stochastic inverse/optimal control problems for large-scale PDE models
such as those of poromechanics is intractable using current methods,
which suffer from the ``curse of dimensionality.''  Thus, our
challenge is to overcome these barriers by developing scalable methods
and algorithms that exploit the problem structure to reduce effective
dimensionality. Overcoming the scalability problem by exploiting
low-dimensional problem structure is essential if we are to have any
hope of optimally and sustainably managing natural and human resources
based on predictive physics-based models, which is a central goal of
the emerging field of computational sustainability.

{\bf Intellectual merits.}
%
The proposed work will result in a rational, physics-model-based,
end-to-end computational sustainability framework that accounts for
uncertainty across all components: from observations, through inverse
solution, through optimal design of observation locations, to
model-predictive optimal control for sustainable \CO2 storage that
reduces the risk of seismicity at existing faults. By exploiting
problem structure to reduce effective problem dimension, we expect
that we will be able to overcome the curse of dimensionality that
plagues PDE-based inversion and control under uncertainty, leading to
significantly improved methods for inference of subsurface models from
data, and therefore the sustainable management of geological \CO2
storage.

{\bf Broader impacts.}
%
Dovetailing with our research activities, we propose an educational
and outreach program designed to communicate to a more general audience of
students, disciplinary researchers, and the lay public the fruits of
our work, and of the wider benefits of the integration of computer \&
computational science, computational statistics, applied math,
engineering, and geosciences to address sustainability problems.
%
We will parlay our leadership positions in our respective communities
to organize workshops and international meetings, edit volumes, 
% teach summer schools, 
develop courses, create educational programs, and engage in community
outreach activities---as we have done in the past---but with greater
emphasis on the field of computational sustainability. Beyond the
CCS sustainability setting,
% the framework we develop is
% clearly applicable to a much broader set of science and engineering
% problems for which large-scale uncertain models must be inferred from
% large-scale uncertain data, and then used to solve decision-making
% problems under uncertainty.  Thus our work will impact a much broader
% set of problems than those targeted here.
{\em the framework we develop is clearly applicable to a much broader
  set of science and engineering problems for which large-scale
  uncertain models must be inferred from large-scale uncertain data,
  and then used to solve optimal decision-making problems under
  uncertainty}.
% the trick is the design of problem-specific reduction operators.

\end{document}


